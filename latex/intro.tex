En el presente Trabajo Práctico diseñamos e implementamos una base de datos para el Campeonato Mundial de Taekwon-do ITF. Esta
base de datos permite la inscripción de competidores, en las diferentes modalidades y categorías, para escuelas de todo el mundo.
Para realizar esta labor construimos un Modelo de Entidad Relación(MER) y lo implementamos utilizando el motor de base
de datos PostgreSQL. A lo largo del presente informe detallaremos las distintas partes del MER (diagrama entidad relación, restricciones,
modelo relacional) y mostraremos la documentación del diseño físico de la base de datos junto con el código de la misma.

\subsection{Asunciones}
\begin{enumerate}
	\item Consideramos que los coeaches son alumnos.
	\item Asumimos que un competidor se puede inscribir aunque no cumpla los requerimientos de la categoría, en este caso, el competidor no estará habilitado y no podrá competir o ganar ningún puesto en el podio.
	\item Asumimos que cada ring tiene un solo grupo de árbitros.
	\item Solo consideramos los tres primeros ganadores de cada competencia y no hacemos ninguna referencia a las estructuras de las llaves.
	\item Asumimos que hay tres categorías de edades ``Juveniles'', ``adultos'' y ``Veteranos'' con sus correspondientes rangos (14 a 17, 18 a 35, 36 en adelante).
	\item Asumimos que el valor asociado al peso de una categoría es el peso máximo que puede tener un competidor para estar habilitado en dicha categoría.
\end{enumerate}