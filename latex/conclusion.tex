El diseño y construcción de la base de datos para el Campeonato Mundial de Taekwon-do ITF, resulto en un desafío interesante
para comprobar la efectividad de los modelos y conceptos aprendidos en la materia.
 A lo largo del trabajo práctico nos vimos forzados a repensar varias veces algunas de las partes de nuestro modelo de entidad relación. Algunas de estas fueron la representación del alumno como
competidor individual y como parte de un equipo, la forma en que se representaban las competencias y su relación tanto con
sus competidores como con los ganadores de los primeros tres puestos y la manera en que se asociaban los distintos tipos de
árbitros con los Rings donde se realizaban las competencias. Todos estos problemas aparecieron durante el armado del diagrama de entidad
relación y, por lo tanto, no se necesito realizar cambios demasiado complicados en la base de datos durante su implementación. El diseño de la base de datos resulto simple gracias al modelo
relacional extraído a partir del MER y, a su vez, las restricciones de los mismos fueron muy útiles
para definir los triggers que se necesitaban para garantizar el correcto manejo de los datos. Podríamos concluir que, a pesar de las dificultades que surgieron,
el modelo de entidad relación simplifica enormemente la implementación de una base de datos y dificulta la aparición de
errores graves de diseño a lo largo de la misma.
