En esta sección están especificadas todas las restricciones adicionales del MER y MR introducidos en las secciones anteriores.

\subsection{Restricciónes provenientes del MER:}
\begin{enumerate}
    \item Todo coach no puede acompañar más de 5 inscripciones de competidores individuales.
    \item Todos los competidores que pertenecen a un mismo equipo son inscriptos por la misma escuela.
    \item Todo equipo está formado por cinco competidores con el atributo \textit{``Titular''} en 1 indicando que son titulares del equipo.
    \item Todo equipo esta formado por al menos tres competidores con el atributo \textit{``Titular''} en 0 indicando que son suplentes en el equipo.
    \item Todo competidor pertenece a un equipo (IdEquipo distinto de \textit{NULL}) si y solo si el atributo \textit{``Titular''} es distinto de \textit{NULL}.
    \item Todo competidor inscripto en una competencia es acompañado por un coach enviado por la misma escuela que lo inscribió.
    \item Todo equipo inscripto en una competencia es acompañado por un coach enviado por la misma escuela que inscribió a los competidores del equipo.
    \item Ningún competidor puede ser acompañado en una competencia por un coach que tenga el mismo \textit{``DNI''} que el.
    \item Ningún equipo puede ser acompañado en una competencia por un coach que tenga el mismo '\textit{``DNI''} que alguno de sus integrantes.
    \item Todas las competencias que tengan atributo \textit{``Graduacion''} deben ser realizadas en rings donde el juez de mesa sea un árbitro con graduación mayor a la de la competencia.
    \item Todas las competencias que tengan atributo \textit{``Graduacion''} deben ser realizadas en rings donde todos los jueces sean árbitros con graduación mayor a la de la competencia.
    \item Todas las competencias que tengan atributo \textit{``Graduacion''} deben ser realizadas en rings donde el árbitro central sea un árbitro con graduación mayor a la de la competencia.
    \item Todas las competencias que tengan atributo \textit{``Graduacion''} deben ser realizadas en rings donde todos los árbitros de recambio sean árbitros con graduación mayor a la de la competencia.
    \item Todo ring tiene al menos tres arbitros de recambio.
    \item Ningún competidor puede estar en más de una de las siguientes relaciones: \textit{``PrimerLugarEn''}, \textit{``SegundoLugarEn''}, \textit{``TercerLugarEn''} para una misma competencia.
    \item Ningún equipo puede estar en más de una de las siguientes relaciones: \textit{``PrimerLugarEn''}, \textit{``SegundoLugarEn''}, \textit{``TercerLugarEn''} para una misma competencia.
    \item Todo competidor que está en alguna de las relaciones \textit{``PrimerLugarEn''}, \textit{``SegundoLugarEn''} o \textit{``TercerLugarEn''} debe estar inscripto y habilitado en dicha competencia a la cual pertenece la relación.
    \item Todo Equipo que está en alguna de las relaciones \textit{``PrimerLugarEn''}, \textit{``SegundoLugarEn''} o \textit{``TercerLugarEn''} debe estar inscripto y habilitado en dicha competencia a la cual pertenece la relación.
    \item Cada competidor puede estar inscripto en una sola categoria de cada modalidad.
    \item Todos los competidores que pertenecen a un mismo equipo son del mismo sexo.
\end{enumerate}

\subsection{Restricciones de las representaciones de los datos:}
\begin{itemize}
    \item El atributo \textit{``Graduacion''} de la entidad Alumno tiene que contener valores entre 1 y 6 inclusive.
    \item El atributo \textit{``Sexo''} de cualquier entidad siempre toma valores ``F'' para ``Femenino'' o ``M'' para ``Masculino''.
    \item El atributo \textit{``Edad''} de cualquier entidad siempre toma valores ``Juveniles'', ``Adultos'' o ``Veteranos''.
    \item El atributo \textit{``Peso''} de cualquier entidad siempre toma valores entre 1 y 300.
    \item El atributo \textit{``FechaDeNacimiento''} de la entidad competidor debe contener una fecha valida (previa al día actual).
    \item El atributo \textit{``Titular''} de la entidad competidor siempre toma alguno de los siguientes valores:
    \begin{itemize}
        \item 0: Suplente 
        \item 1: Titular.
    \end{itemize} 
    \item El atributo \textit{``TipoCompetencia''} de la entidad competencia siempre toma alguno de los siguientes valores: 
    \begin{itemize}
        \item 0: CompetenciaIndividual
        \item 1: CompetenciaCombateEquipos
    \end{itemize} 
    \item El atributo \textit{``Modalidad''} de la entidad CombateIndividual siempre toma alguno de los siguientes valores:
    \begin{itemize}
        \item 0: CompetenciaSalto
        \item 1: CompetenciaFormas
        \item 2: CompetenciaCombateIndividual
        \item 3: CompetenciaRotura
    \end{itemize}
\end{itemize}

\subsection{Restricciones provenientes del MR:}
\begin{itemize}
    \item \textbf{Herencias de Alumno:}
    \begin{itemize}
        \item Competidor.DNI debe estar en Alumno.DNI.
        \item Coach.DNI debe estar en Alumno.DNI.
        \item Alumno.DNI debe estar en Competidor.DNI o (no excluyente) Coach.DNI.
    \end{itemize}
    \item \textbf{Herencias de Competencia:}
    \begin{itemize}
        \item CompetenciaCombateEquipo.IdCompetencia debe estar en Competencia.IdCompetencia
        \item CompetenciaIndividual.IdCompetencia debe estar en Competencia.IdCompetencia
        \item Competencia.IdCompetencia debe estar en CompetenciaCombateEquipo.IdCompetencia o (excluyente) en CompetenciaIndividual.IdCompetencia.
    \end{itemize}
    \item \textbf{Herencias de CompetenciaIndividual:}
    \begin{itemize}
        \item CompetenciaSalto.IdCompetencia debe estar en CompetenciaIndividual.IdCompetencia
        \item CompetenciaFormas.IdCompetencia debe estar en CompetenciaIndividual.IdCompetencia
        \item CompetenciaRotura.IdCompetencia debe estar en CompetenciaIndividual.IdCompetencia
        \item CompetenciaCombateIndividual.IdCompetencia debe estar en CompetenciaIndividual.IdCompetencia
        \item CompetenciaIndividual.IdCompetencia debe estar en CompetenciaSalto.IdCompetencia o (excluyente) en CompetenciaFormas.IdCompetencia o (excluyente) en CompetenciaRotura.IdCompetencia o (excluyente) en CompetenciaCombateIndividual.IdCompetencia.
    \end{itemize}
    \item \textbf{Herencias de Árbitro:}
    \begin{itemize}
        \item PresidenteDeMesa.NroDePlacaDeÁrbitro debe estar en Árbitro.NroDePlacaDeÁrbitro
        \item Juez.NroDePlacaDeÁrbitro debe estar en Árbitro.NroDePlacaDeÁrbitro
        \item ÁrbitroCentral.NroDePlacaDeÁrbitro debe estar en Árbitro.NroDePlacaDeÁrbitro
        \item ÁrbitroDeRecambio.NroDePlacaDeÁrbitro debe estar en Árbitro.NroDePlacaDeÁrbitro
        \item Arbitro.NroDePlacaDeÁrbitro debe estar en PresidenteDeMesa.NroDePlacaDeÁrbitro o (excluyente) en Juez.NroDePlacaDeÁrbitro o (excluyente) ÁrbitroCentral.NroDePlacaDeÁrbitro o (excluyente) en ÁrbitroDeRecambio.NroDePlacaDeÁrbitro.
    \end{itemize}
    \item \textbf{Claves foráneas:}
    \begin{itemize}
        \item Maestro.IdPais debe estar en Pais.IdPais.
        \item Escuela.IdPais debe estar en Pais.IdPais.
        \item Escuela.NroDePlacaDeInstructor debe estar en Maestro.NroDePlacaDeInstructor.
        \item Competidor.IdEscuela debe estar en Escuela.IdEscuela.
        \item Escuela.IdEquipo debe estar en Equipo.IdEquipo o ser NULL.
        \item Coach.IdEscuela debe estar en Coach.IdEscuela.
        \item Árbitro.IdPais debe estar en Pais.IdPais.
        \item CompetenciaIndividual.PrimerLugar debe estar en Competidor.DNI.
        \item CompetenciaIndividual.SegundoLugar debe estar en Competidor.DNI.
        \item CompetenciaIndividual.TercerLugar debe estar en Competidor.DNI.
        \item CompetenciaCombateEquipo.PrimerLugar debe estar en Competidor.DNI.
        \item CompetenciaCombateEquipo.SegundoLugar debe estar en Competidor.DNI.
        \item CompetenciaCombateEquipo.TercerLugar debe estar en Competidor.DNI.
        \item SeRealizaEn.IdCompetencia debe estar en Competencia.IdCompetencia.
        \item SeRealizaEn.IdRing debe estar en Ring.IdRing.
        \item InscriptoEn.IdCompetencia debe estar en Competencia.IdCompetencia.
        \item InscriptoEn.IdCompetidor debe estar en Competidor.DNI.
        \item InscriptoEn.IdCoach debe estar en Coach.DNI.
        \item InscriptoEn.IdCompetencia debe estar en Competencia.IdCompetencia.
        \item InscriptoEn.IdEquipo debe estar en Equipo.IdEquipo.
        \item InscriptoEn.IdCoach debe estar en Coach.DNI.
    \end{itemize}
\end{itemize}
